%-------------------------------------------------------------------------------
%	SECTION TITLE
%-------------------------------------------------------------------------------
\cvsection{Projects}
%-------------------------------------------------------------------------------
%	CONTENT
%-------------------------------------------------------------------------------
\begin{cventries}
%---------------------------------------------------------
  \cventry
    {The University of Edinburgh (Supervised by Liam O'Connor)} % Affiliation/role
    {Holbert Proof Assistant (Honours Project)} % Organization/group
    {Edinburgh, SCOTLAND} % Location
    {Sept. 2021 - April 2022} % Date(s)
    {
      \color{awesome}\autour{\textbf{HASKELL}} \autour{\textbf{HTML}} \autour{\textbf{CSS}}\color{graytext}\ \ \ \href{https://github.com/chrisjpm/holbert}{\faGithub\acvHeaderIconSep\@chrisjpm/\@holbert}\ \ \ \href{http://liamoc.net/holbert}{\faGlobe\acvHeaderIconSep\@Holbert Demo}\ \ \ \href{https://github.com/chrisjpm/holbert/blob/master/report.pdf}{\faFile*[regular]\acvHeaderIconSep\@Thesis}
      \vspace{1.6em}
      \begin{cvitems} % Description(s) of experience/contributions/knowledge
        \item Holbert is a proof assistant for students with no prior experience in automated reasoning. It aims to be used in the classroom to aid the teaching of the foundations of programming languages. I enabled applying rules by elimination, and writing proofs by induction and cases.
        \item Grade: \textbf{76\% (A3)}
      \end{cvitems}
    }
    
%---------------------------------------------------------
%   \cventry
%     {The University of Edinburgh} % Affiliation/role
%     {Information Retrieval} % Organization/group
%     {Edinburgh, SCOTLAND} % Location
%     {Sept. 2021 - Present} % Date(s)
%     {
%       \color{awesome}\autour{\textbf{PYTHON}}\color{graytext}\ \ \ \textit{GitHub repos private, available upon request}
%       \vspace{1.6em}
%       \begin{cvitems} % Description(s) of experience/contributions/knowledge
%         \item Pre-processes text and creates a positional inverted index. Using the index, performs boolean searches, phrase searches, proximity searches and ranked IR based on TF-IDF. Also conducts IR evaluation, text analysis and text classification.
%         \item Grade: \textbf{95.67\% (A1)}
%       \end{cvitems}
%     }

%---------------------------------------------------------
  % \cventry
  %   {The University of Edinburgh} % Affiliation/role
  %   {Search Engine for YouTube Captions} % Organization/group
  %   {Edinburgh, SCOTLAND} % Location
  %   {Jan. 2022 - March 2022} % Date(s)
  %   {
  %     \color{awesome}\autour{\textbf{HTML}} \autour{\textbf{CSS/SCSS}} \autour{\textbf{TYPESCRIPT}} \autour{\textbf{IONIC}}\color{graytext}\ \ \ \href{https://github.com/chrisjpm/YouTube-Caption-Search-Engine}{\faGithub\acvHeaderIconSep\@chrisjpm/\@Swimming-Relay-Order-Calculator}\ \ \ \href{https://clipsninja.netlify.app/home}{\faGlobe\acvHeaderIconSep\@Website}
  %     \vspace{1.6em}
  %     \begin{cvitems} % Description(s) of experience/contributions/knowledge
  %       \item Clips Ninja is a search engine for YouTube captions and returns relevant timestamps to watch from. I worked on the front-end
  %     \end{cvitems}
  %   }
    
%---------------------------------------------------------
  % \cventry
  %   {The University of Edinburgh} % Affiliation/role
  %   {`Learn' Course Web-Page Redesign} % Organization/group
  %   {Edinburgh, SCOTLAND} % Location
  %   {Sept. 2021 - Dec. 2021} % Date(s)
  %   {
  %     \color{awesome}\autour{\textbf{FIGMA}}\color{graytext}\ \ \ \href{https://www.figma.com/proto/Q6wnTtZs5rP21QqnWcJBP6/G42-CW3?node-id=0\%3A1}{\faFigma\acvHeaderIconSep\@Prototype}
  %     \vspace{1.6em}
  %     \begin{cvitems} % Description(s) of experience/contributions/knowledge
  %       \item An improved UI/UX for viewing grades on a course web-page, created in Figma with partial interactivity.
  %       \item Grade: \textbf{70\% (A3)}
  %     \end{cvitems}
  %   }
    
%---------------------------------------------------------
  % \cventry
  %   {The University of Edinburgh} % Affiliation/role
  %   {System Design Project} % Organization/group
  %   {Edinburgh, SCOTLAND} % Location
  %   {Jan. 2021 - April 2021} % Date(s)
  %   {
  %     \color{awesome}\autour{\textbf{HTML}} \autour{\textbf{CSS}} \autour{\textbf{JAVASCRIPT}} \autour{\textbf{FIGMA}} \autour{\textbf{PHOTOSHOP CS6}} \autour{\textbf{PREMIERE PRO CS6}}\color{graytext}\ \ \ \href{https://github.com/DeliverED-Home}{\faGithub\acvHeaderIconSep\@DeliverED-Home}\ \ \ \href{https://DeliverED-Home.github.io/DeliverED-Site}{\faGlobe\acvHeaderIconSep\@Product Website}
  %     \vspace{1.6em}
  %     \begin{cvitems} % Description(s) of experience/contributions/knowledge
  %       \item In a group of eight students, we were tasked to design and implement a complete system to solve some practical and useful problem.
  %       \item I designed all graphics and made the website to demonstrate our product. I also conducted usability testing for our Android app using a Figma prototype and a scripted interview with five participants.
  %       % \item Grade: \textbf{68\% (B)}
  %     \end{cvitems}
  %   }
    
%---------------------------------------------------------
  \cventry
    {The University of Edinburgh} % Affiliation/role
    {Informatics Large Practical} % Organization/group
    {Edinburgh, SCOTLAND} % Location
    {Sept. 2020 - Dec. 2020} % Date(s)
    {
      \color{awesome}\autour{\textbf{JAVA}} \autour{\textbf{GEOJSON}}\color{graytext}\ \ \ \href{https://github.com/chrisjpm/inf3-ilp-cw1}{\faGithub\acvHeaderIconSep\@chrisjpm/\@inf3-ilp-cw1}\ \ \ \href{https://github.com/chrisjpm/inf3-ilp-cw2}{\faGithub\acvHeaderIconSep\@chrisjpm/\@inf3-ilp-cw2}
      \vspace{1.6em}
      \begin{cvitems} % Description(s) of experience/contributions/knowledge
        \item A heat map to visualise the predictions of air quality sensor readings, partitioned into a regular 10×10 grid. And an application which calculates a flight path which visits as many of the sensors listed for that date as possible.
        \item Grade: \textbf{95\% (A1)}
      \end{cvitems}
    }
%---------------------------------------------------------
  \cventry
    {Shawlands Academy} % Affiliation/role
    {Swimming Relay Order Calculator} % Organization/group
    {Glasgow, SCOTLAND} % Location
    {Oct. 2017 - June 2018} % Date(s)
    {
      \color{awesome}\autour{\textbf{HTML}} \autour{\textbf{CSS}} \autour{\textbf{JAVASCRIPT}} \autour{\textbf{NODE.JS}} \autour{\textbf{EXPRESS}} \autour{\textbf{HANDLEBARS}} \autour{\textbf{MYSQL}} \autour{\textbf{GOOGLE CLOUD}} \autour{\textbf{HEROKU}}\color{graytext}\ \ \ \href{https://github.com/chrisjpm/Swimming-Relay-Order-Calculator}{\faGithub\acvHeaderIconSep\@chrisjpm/\@Swimming-Relay-Order-Calculator}
      \vspace{1.6em}
      \begin{cvitems} % Description(s) of experience/contributions/knowledge
        \item Enters swimmers' details and personal best times through a form and select the requirements of a relay team. A report is generated on different combinations of swimmers and strokes with their total time sorted fastest to slowest.
        \item Grade: \textbf{100\% (A1)}
      \end{cvitems}
    }
%---------------------------------------------------------
\end{cventries}
\vspace{-.5em}
