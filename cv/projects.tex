%-------------------------------------------------------------------------------
%	SECTION TITLE
%-------------------------------------------------------------------------------
\cvsection{Projects}


%-------------------------------------------------------------------------------
%	CONTENT
%-------------------------------------------------------------------------------

\begin{cventries}
%---------------------------------------------------------
  \cventry
    {The University of Edinburgh} % Affiliation/role
    {Nicer Proofs By Induction in the Holbert Proof Assistant (Honours Project)} % Organization/group
    {Edinburgh, SCOTLAND} % Location
    {Sept. 2021 - Present} % Date(s)
    {
      \color{awesome}\autour{\textbf{HASKELL}} \autour{\textbf{HTML}}\color{graytext}\ \ \ \href{https://github.com/chrisjpm/holbert}{\faGithub\acvHeaderIconSep\@chrisjpm/\@holbert}\ \ \ \href{http://liamoc.net/holbert}{\faGlobe\acvHeaderIconSep\@Holbert Demo}%\ \ \ \href{}{\faFile*[regular]\acvHeaderIconSep\@Thesis}
      \vspace{1.8em}
      \begin{cvitems} % Description(s) of experience/contributions/knowledge
        \item Holbert is a web-based, interactive proof assistant built on higher-order logic and natural deduction. It is written in Haskell and rendered in-browser with the Miso front-end framework.
        \item Holbert’s primary focus is to be an educational tool for students and promote teaching of the foundations of programming languages for users with no prior experience in automated reasoning.
        \item My contribution to this project is implementing features to improve proofs by induction.
      \end{cvitems}
    }
    \vspace{.08cm}
    
%---------------------------------------------------------
  \cventry
    {The University of Edinburgh} % Affiliation/role
    {Text Technologies for Data Science} % Organization/group
    {Edinburgh, SCOTLAND} % Location
    {Sept. 2021 - Present} % Date(s)
    {
      \color{awesome}\autour{\textbf{PYTHON}}\color{graytext}\ \ \ \textit{GitHub repos private, available upon request}
      \vspace{1.8em}
      \begin{cvitems} % Description(s) of experience/contributions/knowledge
        \item Assignment 1: A simple IR tool that pre-processes text, creates a positional inverted index. Using the positional inverted index, perform boolean search, phrase search, proximity search and ranked IR based on TFIDF.
        \item Assignment 2: IR evaluation, text analysis and text classification.
        \item Grade: \textbf{95.67\% (A1)} (In progress)
      \end{cvitems}
    }
    \vspace{.08cm}
%---------------------------------------------------------

  \cventry
    {The University of Edinburgh} % Affiliation/role
    {Human Computer Interaction} % Organization/group
    {Edinburgh, SCOTLAND} % Location
    {Sept. 2021 - Dec. 2021} % Date(s)
    {
      \color{awesome}\autour{\textbf{FIGMA}}\color{graytext}\ \ \ \href{https://www.figma.com/proto/SEoaoC4LAWCmH1iGFA4AEO/G42-CW1?node-id=0\%3A1}{\faFigma\acvHeaderIconSep\@Prototype 1}\ \ \ \href{https://www.figma.com/proto/Q6wnTtZs5rP21QqnWcJBP6/G42-CW3?node-id=0\%3A1}{\faFigma\acvHeaderIconSep\@Prototype 2}
      \vspace{1.8em}
      \begin{cvitems} % Description(s) of experience/contributions/knowledge
        \item Create an interactive prototype that improves the UX of Edinburgh University's 'Learn' site for certain tasks.
        \item Grade: \textbf{68\% (B)}
      \end{cvitems}
    }
    \vspace{.08cm}
%---------------------------------------------------------

  \cventry
    {The University of Edinburgh} % Affiliation/role
    {System Design Project} % Organization/group
    {Edinburgh, SCOTLAND} % Location
    {Jan. 2021 - April 2021} % Date(s)
    {
      \color{awesome}\autour{\textbf{HTML}} \autour{\textbf{CSS}} \autour{\textbf{JAVASCRIPT}} \autour{\textbf{FIGMA}} \autour{\textbf{PHOTOSHOP CS6}} \autour{\textbf{PREMIERE PRO CS6}}\color{graytext}\ \ \ \href{https://github.com/DeliverED-Home}{\faGithub\acvHeaderIconSep\@DeliverED-Home}\ \ \ \href{https://DeliverED-Home.github.io/DeliverED-Site}{\faGlobe\acvHeaderIconSep\@Product Website}
      \vspace{1.8em}
      \begin{cvitems} % Description(s) of experience/contributions/knowledge
        \item In a group of nine students, we were tasked to design and implement a complete system to solve some practical and useful problem. This entailed a convincing demonstration of a potential product, suitable for presentation to a client/investor.
        \item I was the co-product manager and graphic designer. This wrote and edited reports, created four demo videos and made the final website to demonstrate our product.
        \item I conducted usability testing for our Android app using a prototype created with Figma and ran a scripted interview with five participants.
        \item Grade: \textbf{68\% (B)}
      \end{cvitems}
    }
    \vspace{.08cm}
%---------------------------------------------------------

  \cventry
    {The University of Edinburgh} % Affiliation/role
    {Informatics Large Practical} % Organization/group
    {Edinburgh, SCOTLAND} % Location
    {Sept. 2020 - Dec. 2020} % Date(s)
    {
      \color{awesome}\autour{\textbf{JAVA}} \autour{\textbf{GEOJSON}}\color{graytext}\ \ \ \href{https://github.com/chrisjpm/inf3-ilp-cw1}{\faGithub\acvHeaderIconSep\@chrisjpm/\@inf3-ilp-cw1}\ \ \ \href{https://github.com/chrisjpm/inf3-ilp-cw2}{\faGithub\acvHeaderIconSep\@chrisjpm/\@inf3-ilp-cw2}
      \vspace{1.8em}
      \begin{cvitems} % Description(s) of experience/contributions/knowledge
        \item Assignment 1: A heat map to visualise the predictions of air quality sensor readings, partitioned into a regular 10x10 grid.
        \item Assignment 2: Develop an application which calculates a flight path which visits as many of the sensors listed for that date as possible. The application produces a report on the drone’s flight as a Geo-JSON map and a log file.
        \item Grade: \textbf{95\% (A1)}
      \end{cvitems}
    }
    \vspace{.08cm}
%---------------------------------------------------------
\end{cventries}
