%-------------------------------------------------------------------------------
%	SECTION TITLE
%-------------------------------------------------------------------------------
\cvsection{Projects}
%-------------------------------------------------------------------------------
%	CONTENT
%-------------------------------------------------------------------------------
\begin{cventries}
%---------------------------------------------------------
  \cventry
    {The University of Edinburgh (Supervised by Liam O'Connor)} % Affiliation/role
    {Honours Project - Nicer Proofs By Induction in the Holbert Proof Assistant} % Organization/group
    {Edinburgh, SCOTLAND} % Location
    {Sept. 2021 - Present} % Date(s)
    {
      \color{awesome}\autour{\textbf{HASKELL}} \autour{\textbf{HTML}}\color{graytext}\ \ \ \href{https://github.com/chrisjpm/holbert}{\faGithub\acvHeaderIconSep\@chrisjpm/\@holbert}\ \ \ \href{http://liamoc.net/holbert}{\faGlobe\acvHeaderIconSep\@Holbert Demo}%\ \ \ \href{}{\faFile*[regular]\acvHeaderIconSep\@Thesis}
      \vspace{1.6em}
      \begin{cvitems} % Description(s) of experience/contributions/knowledge
        \item Holbert is a web-based, interactive proof assistant built on higher-order logic and natural deduction. It is written in Haskell and rendered in-browser with the Miso front-end framework.
        \item Holbert’s primary focus is to be an educational tool for students and promote teaching of the foundations of programming languages for users with no prior experience in automated reasoning.
        \item My contribution to this project is implementing features to improve proofs by induction.
      \end{cvitems}
    }
    
%---------------------------------------------------------
  \cventry
    {The University of Edinburgh} % Affiliation/role
    {Text Technologies for Data Science - Information Retrieval} % Organization/group
    {Edinburgh, SCOTLAND} % Location
    {Sept. 2021 - Present} % Date(s)
    {
      \color{awesome}\autour{\textbf{PYTHON}}\color{graytext}\ \ \ \textit{GitHub repos private, available upon request}
      \vspace{1.6em}
      \begin{cvitems} % Description(s) of experience/contributions/knowledge
        \item Assignment 1: A simple IR tool that pre-processes text, creates a positional inverted index. Using the positional inverted index, perform boolean search, phrase search, proximity search and ranked IR based on TFIDF.
        \item Assignment 2: IR evaluation, text analysis and text classification.
        \item Grade: \textbf{95.67\% (A1)} (In progress)
      \end{cvitems}
    }
    
%---------------------------------------------------------
  \cventry
    {The University of Edinburgh} % Affiliation/role
    {Human Computer Interaction - Learn Design} % Organization/group
    {Edinburgh, SCOTLAND} % Location
    {Sept. 2021 - Dec. 2021} % Date(s)
    {
      \color{awesome}\autour{\textbf{FIGMA}}\color{graytext}\ \ \ \href{https://www.figma.com/proto/SEoaoC4LAWCmH1iGFA4AEO/G42-CW1?node-id=0\%3A1}{\faFigma\acvHeaderIconSep\@Prototype 1}\ \ \ \href{https://www.figma.com/proto/Q6wnTtZs5rP21QqnWcJBP6/G42-CW3?node-id=0\%3A1}{\faFigma\acvHeaderIconSep\@Prototype 2}
      \vspace{1.6em}
      \begin{cvitems} % Description(s) of experience/contributions/knowledge
        \item Create an interactive prototype that improves the UX of Edinburgh University's 'Learn' site for certain tasks.
        \item Grade: \textbf{68\% (B)}
      \end{cvitems}
    }
    
%---------------------------------------------------------
  \cventry
    {The University of Edinburgh} % Affiliation/role
    {System Design Project - DeliverED Home} % Organization/group
    {Edinburgh, SCOTLAND} % Location
    {Jan. 2021 - April 2021} % Date(s)
    {
      \color{awesome}\autour{\textbf{HTML}} \autour{\textbf{CSS}} \autour{\textbf{JAVASCRIPT}} \autour{\textbf{FIGMA}} \autour{\textbf{PHOTOSHOP CS6}} \autour{\textbf{PREMIERE PRO CS6}}\color{graytext}\ \ \ \href{https://github.com/DeliverED-Home}{\faGithub\acvHeaderIconSep\@DeliverED-Home}\ \ \ \href{https://DeliverED-Home.github.io/DeliverED-Site}{\faGlobe\acvHeaderIconSep\@Product Website}
      \vspace{1.6em}
      \begin{cvitems} % Description(s) of experience/contributions/knowledge
        \item In a group of nine students, we were tasked to design and implement a complete system to solve some practical and useful problem. This entailed a convincing demonstration of a potential product, suitable for presentation to a client/investor.
        \item I was the co-product manager and graphic designer. I wrote and edited 7 reports, created 5 videos, designed all graphics and made the website to demonstrate our product. I also conducted usability testing for our Android app using a Figma prototype and a scripted interview with five participants.
        \item Grade: \textbf{68\% (B)}
      \end{cvitems}
    }
    
%---------------------------------------------------------
  \cventry
    {The University of Edinburgh} % Affiliation/role
    {Informatics Large Practical - Air Quality Drone} % Organization/group
    {Edinburgh, SCOTLAND} % Location
    {Sept. 2020 - Dec. 2020} % Date(s)
    {
      \color{awesome}\autour{\textbf{JAVA}} \autour{\textbf{GEOJSON}}\color{graytext}\ \ \ \href{https://github.com/chrisjpm/inf3-ilp-cw1}{\faGithub\acvHeaderIconSep\@chrisjpm/\@inf3-ilp-cw1}\ \ \ \href{https://github.com/chrisjpm/inf3-ilp-cw2}{\faGithub\acvHeaderIconSep\@chrisjpm/\@inf3-ilp-cw2}
      \vspace{1.6em}
      \begin{cvitems} % Description(s) of experience/contributions/knowledge
        \item Assignment 1: A heat map to visualise the predictions of air quality sensor readings, partitioned into a regular 10×10 grid.
        \item Assignment 2: Develop an application which calculates a flight path which visits as many of the sensors listed for that date as possible. The application produces a report on the drone’s flight as a Geo-JSON map and a log file.
        \item Grade: \textbf{95\% (A1)}
      \end{cvitems}
    }
%---------------------------------------------------------
  \cventry
    {Shawlands Academy} % Affiliation/role
    {Advanced Higher Computer Science - Swimming Relay Order Calculator} % Organization/group
    {Glasgow, SCOTLAND} % Location
    {Oct. 2017 - June 2018} % Date(s)
    {
      \color{awesome}\autour{\textbf{HTML}} \autour{\textbf{CSS}} \autour{\textbf{JAVASCRIPT}} \autour{\textbf{NODE.JS}} \autour{\textbf{EXPRESS}} \autour{\textbf{HANDLEBARS}} \autour{\textbf{MYSQL}} \autour{\textbf{GOOGLE CLOUD}} \autour{\textbf{HEROKU}}\color{graytext}\ \ \ \href{https://github.com/chrisjpm/Swimming-Relay-Order-Calculator}{\faGithub\acvHeaderIconSep\@chrisjpm/\@Swimming-Relay-Order-Calculator}
      \vspace{1.6em}
      \begin{cvitems} % Description(s) of experience/contributions/knowledge
        \item The user can enter swimmers' details and personal best times via completing a form on the site and then detail the requirements of a relay team, select which swimmers they want to include (by age and gender). They'll receive a report on different combinations of swimmers and strokes with their total time (sorted fastest to slowest). The report can be downloaded as a CSV.
        \item Grade: \textbf{100\% (A1)}
      \end{cvitems}
    }
%---------------------------------------------------------
\end{cventries}
\vspace{-.5em}
